\documentclass{beamer}

\usepackage{amsmath}
\usepackage{graphicx}
\usepackage{listings}

\begin{document}

\title{Intro to blockchain}
% \subtitle{With the bitcoin application}
\author{Zhehao Wang}
\date{Oct 2018}

\frame{\titlepage}

\begin{frame}
\frametitle{Symmetric cryptography}

Consider symmetric cryptoraphy, where

\begin{figure}
  \centering
  \includegraphics[width=\textwidth]{symmetric-encryption}
\end{figure}

The problem: how do I securely transfer \textcolor{blue}{$key~K$} over the network?

\end{frame}

\begin{frame}
\frametitle{Asymmetric cryptography, RSA algorithm}

Find 3 very large positive integers $e$, $d$, $n$ s.t.
$$
(m^e)^d \equiv m ~ (\text{mod} ~ n), ~ ~ \forall m,~ 0 \leq m \le n
$$
Knowing $e$, $n$ or even $m$ it's extremely hard to find $d$.

\begin{itemize}
    \item Public key: $n$, $e$
    \item Private key: $n$, $d$
    \item Message: $m$
\end{itemize}

Operations:
\begin{itemize}
    \item Encryption(m): $c = m^e ~ (\text{mod} ~ n)$
    \item Decryption(c): $m' = c^d ~ (\text{mod} ~ n) = (m^e)^d ~ \text{mod} ~ n$, $m' = m ~ (\text{mod} ~ n)$
\end{itemize}

\end{frame}

\begin{frame}
\frametitle{RSA algorithm, example}

\begin{figure}
  \centering
  \includegraphics[width=\textwidth]{asymmetric-encryption}
\end{figure}

Anyone who wants to talk to Bob can retrieve Bob's public key, use it to encrypt the message, and know that only holder of the corresponding private key can decrypt.

\end{frame}

\begin{frame}[fragile]
\frametitle{RSA algorithm in practice}

In practice, key pairs are much longer.
\begin{lstlisting}[language=bash]
  $ ssh-keygen -t rsa -b 4096
\end{lstlisting}

Each key pair corresponds with an \textbf{identity}.

\vspace{0.2in}

The two operations:
\begin{itemize}
    \item {Encryption/Decryption}: Alice uses Bob's public key to encrypt, Bob uses his private key to decrypt
    \item {Signing/Validation}: Alice uses her own private key to sign, others use Alice's public key to verify
\end{itemize}

\vspace{0.2in}
Vs symmetric encryption: more computation, but solves the problem of key transfer (and many others)

\vspace{0.2in}
Related concepts: digital signature, certificate, public key infrastructure

\end{frame}

\begin{frame}
\frametitle{To design cryptocurrency}

Imagine designing a cryptocurrency
\begin{itemize}
    \item An account is a public/private key pair!
    \item If I pay someone (a public key identity), I sign with my private key: others can verify \textit{I} made the payment, and I cannot refute later on
\end{itemize}

But how does anyone know I have enough money left to make the payment? \textit{Double spending}
\begin{itemize}
    \item The \textbf{centralized} way: we all agree on (and preinstall) having one party to trust, who keeps track of everyone's account balances
    \item The \textbf{decentralized} way: is it possible to have \textit{all} of us keep track of account balances together?
\end{itemize}

\end{frame}

\begin{frame}
\frametitle{To design cryptocurrency - cont}

Design considerations:
\begin{itemize}
    \item Distributed
    \item Trustless, no a-prori trust relationship established
    \item Consensus, everyone agrees on account balances
\end{itemize}

\vspace{0.3in}
Why is this problem unique:

\begin{itemize}
\item I don't trust anyone I talk to! (think Raft, dynamo, etc) But we can still agree on something.
\end{itemize}

\end{frame}

\begin{frame}
\frametitle{Distributed trustless consensus - intuition}

Distributed consensus: leadership election, and \textbf{state-machine replication}
(think raft)

\vspace{0.2in}
If we can agree on the entire history of transactions, we would know if someone is trying to double spend.
(think log structured merge tree (e.g. Cassandra) / journaling file system / git)

\vspace{0.2in}
To agree on the history,

\begin{itemize}
    \item what if we assume: there are more peers in the network who are honest, than those who are not
    \item we can have everyone tell everyone else their view of the entire history, and hope they converge...
\end{itemize}

\end{frame}

\begin{frame}
\frametitle{Blockchain in bitcoin whitepaper}

Blockchain makes a different assumption
\begin{itemize}
    \item \textbf{honest peers own more computation power than those who are dishonest}
\end{itemize}

\vspace{0.2in}

Consequently, if it takes computation power to grow the history, then honest peers can grow the history faster than dishonest ones
\begin{itemize}
    \item Peers agree on \textbf{trusting the longest chain}, and grow based on that
    \item In order to grow the history, peers perform a \textbf{proof of work}, which is computationally non-trivial
\end{itemize}

\end{frame}

\begin{frame}
\frametitle{Building a block}

\begin{figure}
  \centering
  \includegraphics[width=0.85\textwidth]{building-a-block}
\end{figure}

\end{frame}

\begin{frame}
\frametitle{Growing a chain}

\end{frame}

\begin{frame}
\frametitle{Proof of work}

\end{frame}

\begin{frame}
\frametitle{Mining and incentive}

\end{frame}

\end{document}
