\documentclass{beamer}

\usepackage{amsmath}
\usepackage{graphicx}
\usepackage{listings}

\begin{document}

\title{Introduction to blockchain}
\subtitle{With bitcoin as an example}
\author{Zhehao Wang}
\date{Oct 2018}

\frame{\titlepage}

\begin{frame}
\frametitle{Symmetric cryptography}

Consider symmetric cryptoraphy, where

\begin{figure}
  \centering
  \includegraphics[width=\textwidth]{symmetric-encryption}
\end{figure}

The problem: how do I securely transfer \textcolor{blue}{$key~K$} over the network?

\end{frame}

\begin{frame}
\frametitle{Asymmetric cryptography, RSA algorithm}

Find 3 very large positive integers $e$, $d$, $n$ s.t.
$$
(m^e)^d \equiv m ~ (\text{mod} ~ n), ~ ~ \forall m,~ 0 \leq m \le n
$$
Knowing $e$, $n$ or even $m$ it's extremely hard to find $d$.

\begin{itemize}
  \item Public key: $n$, $e$
  \item Private key: $n$, $d$
  \item Message: $m$
\end{itemize}

Operations:
\begin{itemize}
    \item Encryption(m): $c = m^e ~ (\text{mod} ~ n)$
    \item Decryption(c): $m' = c^d ~ (\text{mod} ~ n) = (m^e)^d ~ \text{mod} ~ n$, $m' = m ~ (\text{mod} ~ n)$
\end{itemize}

\end{frame}

\begin{frame}
\frametitle{RSA algorithm, example}

\begin{figure}
  \centering
  \includegraphics[width=\textwidth]{asymmetric-encryption}
\end{figure}

Anyone who wants to talk to Bob can retrieve Bob's public key, use it to encrypt the message, and know that only holder of the corresponding private key can decrypt.

\end{frame}

\begin{frame}[fragile]
\frametitle{RSA algorithm in practice}

In practice, key pairs are much longer.
\begin{lstlisting}[language=bash]
  $ ssh-keygen -t rsa -b 4096
\end{lstlisting}

Each key pair corresponds with an \textbf{identity}.

\vspace{0.2in}

The two operations:
\begin{itemize}
  \item {Encryption/Decryption}: Alice uses Bob's public key to encrypt, Bob uses his private key to decrypt
  \item {Signing/Validation}: Alice uses her own private key to sign, others use Alice's public key to verify
\end{itemize}

\vspace{0.2in}
Vs symmetric encryption: more computation, but solves the problem of key transfer (and many others)

\vspace{0.2in}
Related concepts: digital signature, certificate, public key infrastructure

\end{frame}

\begin{frame}
\frametitle{To design a cryptocurrency}

Imagine designing a cryptocurrency
\begin{itemize}
  \item An account is a public/private key pair!
  \item If I pay someone (a public key identity), I sign with my private key: others can verify \textit{I} made the payment, and I cannot repudiate later on
\end{itemize}

But how does anyone know I have enough money left to make the payment? \textit{Double spending}
\begin{itemize}
  \item The \textbf{centralized} way: we all agree on (and preinstall) having one party to trust, who keeps track of everyone's account balances
  \item The \textbf{decentralized} way: is it possible to have \textit{all} of us keep track of account balances together?
\end{itemize}

\end{frame}

\begin{frame}
\frametitle{To design a cryptocurrency - cont}

Considerations:
\begin{itemize}
  \item Distributed
  \item Trustless, no a-prori trust relationship established
  \item Consensus, everyone agrees on account balances
\end{itemize}

\vspace{0.3in}
Why is this problem unique:

\begin{itemize}
  \item I don't trust anyone I talk to
  \item But we can still agree on something
\end{itemize}

\end{frame}

\begin{frame}
\frametitle{Distributed trustless consensus - intuition}

Distributed consensus: leadership election, and \textbf{state-machine replication}\footnote{Think Raft consensus algorithm}

\vspace{0.2in}
If we can agree on the entire history of transactions (state machine), we know if someone is trying to double spend.
\footnote{Think log structured merge tree (e.g. writes in Cassandra) / journaling file system / git}

\vspace{0.2in}
To agree on the history,

\begin{itemize}
  \item what if we assume: there are more peers in the network who are honest, than those who are not
  \item we can have everyone tell everyone else their view of the entire history, and hope they converge...
\end{itemize}

\end{frame}

\begin{frame}
\frametitle{Blockchain, introduction\footnote{As in the classic bitcoin whitepaper, since the field has seen lots of divergents}}

Blockchain makes a different assumption
\begin{itemize}
  \item \textbf{honest peers own more computation power than those who are dishonest}
\end{itemize}

\vspace{0.2in}

Consequently, if it takes computation power to grow the history, then honest peers can grow the history faster than dishonest ones
\begin{itemize}
  \item Peers \textbf{trust the chain with the most computation power invested (i.e. the longest)}, and grow based on that
  \item In order to grow the history, peers perform a \textbf{proof of work}, which is computationally non-trivial
\end{itemize}

\end{frame}

\begin{frame}
\frametitle{Proof of work}

After assembling $X$ transactions in a block\footnote{$X$ depends on the size of transactions, and is usually above 1000 in bitcoin}, find a nonce value to attach to the block, $s.t.$ the hash\footnote{E.g. sha256} of the entire block has an agreed number of $0$s at its end
\begin{itemize}
  \item A block is only considered valid and complete, if one such nonce is attached
  \item No known efficient algorithm exists to find one such value, (mostly) trials
  \item Alternatives exist
\end{itemize}

\vspace{0.2in}
Proof-of-work "verifies" a block (in turn, grows a chain)

\vspace{0.2in}
Peers who carry out proof-of-work are called \textbf{miners}.

\end{frame}

\begin{frame}
\frametitle{Building a block}

\begin{figure}
  \centering
  \includegraphics[width=0.85\textwidth]{building-a-block}
\end{figure}

\end{frame}

\begin{frame}
\frametitle{Growing a chain}

\begin{figure}
  \centering
  \includegraphics[width=0.90\textwidth]{grow-a-chain}
\end{figure}

Genesis block: the start of the chain (e.g. allocating $X$ coins to the creator)

\end{frame}

\begin{frame}
\frametitle{Mining and incentive}

\vspace{0.2in}
Why should anyone verify someone else's transaction (mine)?
\begin{itemize}
  \item Miners are awarded a fixed number of bitcoins for each block mined
  \item Transactions can attach a transaction fee
  \item Winner takes all
\end{itemize}

\vspace{0.2in}
How are new coins introduced to the system?
\begin{itemize}
  \item Genesis block
  \item Mining 'adds' new coins to the system (analogy: miners spend work to find gold nuggets to add to world gold circulation)
\end{itemize}

\end{frame}

\begin{frame}
\frametitle{Operation}

What if two blocks are created at the same time?
\begin{itemize}
  \item A temporary fork. Can introduce an ordering mechanism, or miners may work on different forks, until one gets longer than the other and propagates.
\end{itemize}

\vspace{0.1in}
As the system progresses,
\begin{itemize}
  \item Number of $0$s to satisfy proof-of-work increases \footnote{The system aims to create 1 new block every 10 minutes, and it does so by adjusting the difficulty of proof-of-work}
  \item Mining reward \textbf{halves} every $X$ blocks\footnote{Current reward: 12.5 btc, $X = 210000$. How to incentivize afterwards: transaction fee}
  \item A hard fork could be made for backward incompatible upgrade, or confirmed compromise
\end{itemize}

\vspace{0.1in}
Max number of coins is fixed (inflation-free!)
$$
\lim\limits_{n \to \infty}{\sum_{i = 1}^{n}{\frac{1}{2^i}}} = 1
$$

\end{frame}

\begin{frame}
\frametitle{Tamper resistance and privacy}

Assume dishonest peers work together to produce proof-of-work for a block in which a transaction source account double spends
\begin{itemize}
  \item Dishonest peers have to grow the chain faster than the honest ones, so that the honest ones use their chain (contrary to assumption)
  \item Subverting an existing block in the chain gets exponentially more difficult (grow everything afterwards, and faster)
  \item If you have this much computation power, it'd be more in your interest to mine, rather than to subvert
\end{itemize}

\vspace{0.2in}
Privacy
\begin{itemize}
  \item No tie between account $736f$ with a physical identity
  \item The network does not store anything related with physical identity
\end{itemize}

\end{frame}

\begin{frame}
\frametitle{The trade-offs}

Distributed and trustless come at a cost

\begin{itemize}
  \item Global scale broadcast messages: vs unicast to a bank
  \item Proof-of-work: vs bank keeping records and trusting what they keep
  \item Transaction fee: pay to verifier, or pay to bank
\end{itemize}

\vspace{0.2in}
But also with advantages

\begin{itemize}
  \item Minimal a-priori
  \item Anonymous
  \item Mathematically hard to tamper / counterfeit
\end{itemize}

\end{frame}

\begin{frame}
\frametitle{Analogy to a currency}

\begin{itemize}
  \item Durable
  \item Portable
  \item Divisible
  \item Fungible (like symmetric in an equivalence relation)
  \item Intrinsic value?
\end{itemize}

\end{frame}

\begin{frame}
\frametitle{Extensions - smart contract}

With bitcoin, the \textit{actions} we record on the chain are transactions.

They can be anything else.

\vspace{0.2in}

\begin{itemize}
  \item Casting a vote for a presidential candidate
  \item Publishing a blog post
  \item ...
\end{itemize}

\vspace{0.2in}
How do we express \textit{anything}

The ethereum framework and Solidity programming language.

\end{frame}

\begin{frame}
\frametitle{Summary}

\begin{itemize}
  \item Public key cryptography
  \item Distributed trustless consensus
  \item Proof-of-work and computation power assumption
\end{itemize}

\end{frame}

\begin{frame}
\frametitle{References}

\end{frame}

\end{document}
